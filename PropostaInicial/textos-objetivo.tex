\chapter{Objetivo}
Este projeto visa desenvolver o sistema Turma de Elite, uma aplicação de gerenciamento de aprendizado que visa integrar conceitos de gamificação ao ensino, e deste modo se tornar uma ferramenta que busca auxiliar todo corpo docente e discente de uma instituição escolar.

Para o corpo docente, a ferramenta contará com funcionalidades que permitam acompanhar o desenvolvimento do aluno e modelar desafios, atividades e recompensas.

Para o corpo discente, a aplicação implementará funcionalidades que buscam promover o engajamento dos alunos nas aulas, acrescentando ao processo de realizar exercícios e avaliações, uma dinâmica semelhante aos jogos atuais onde ao completar um desafio, ganham-se recompensas, aumenta-se de nível e desbloqueia novos desafios.

Outro objetivo a ser alcançado será desenvolver a ferramenta que seja flexível às diferentes escolas e que também seja configurável ao ponto de se adaptar tanto para o ensino presencial quanto o remoto. 

Inicialmente o sistema Turma de Elite será desenvolvido para atender a turmas do ensino fundamental II, que compreendem alunos do 6º ao 9º ano, com faixa etária variando entre 11 a 15 anos. Posteriormente, a aplicação poderá ser adaptada para abranger o ensino primário e médio.