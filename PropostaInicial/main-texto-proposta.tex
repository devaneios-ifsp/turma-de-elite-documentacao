%% Adaptado a partir de :
%%    abtex2-modelo-trabalho-academico.tex, v-1.9.2 laurocesar
%% para ser um modelo para os trabalhos no IFSP-SPO

\documentclass[
    % -- opções da classe memoir --
    12pt,               % tamanho da fonte
    openright,          % capítulos começam em pág ímpar (insere página vazia caso preciso)
    %twoside,            % para impressão em verso e anverso. Oposto a oneside
    oneside,
    a4paper,            % tamanho do papel. 
    % -- opções da classe abntex2 --
    %chapter=TITLE,     % títulos de capítulos convertidos em letras maiúsculas
    %section=TITLE,     % títulos de seções convertidos em letras maiúsculas
    %subsection=TITLE,  % títulos de subseções convertidos em letras maiúsculas
    %subsubsection=TITLE,% títulos de subsubseções convertidos em letras maiúsculas
    % Opções que não devem ser utilizadas na versão final do documento
    MODELO,             % indica que é um documento modelo então precisa dos geradores de texto
    TODO,               % indica que deve apresentar lista de pendencias 
    % -- opções do pacote babel --
    english,            % idioma adicional para hifenização
    brazil              % o último idioma é o principal do documento
    ]{ifsp-spo-inf-ctds}

        
% ---

% --- 
% CONFIGURAÇÕES DE PACOTES
% --- 
%\usepackage{etoolbox}
%\usepackage{graphicx}
%\patchcmd{\thebibliography}{\chapter*}{\section*}{}{}


% ---
% Informações de dados para CAPA e FOLHA DE ROSTO
% ---
\titulo{Projeto Turma de Elite}

% Trabalho individual
\autor 
    {
    }
     
%abntex2/wiki/FAQ#como-adicionar-mais-de-um-autor-ao-meu-projeto
\renewcommand{\imprimirautor}{
\begin{tabular}{lr}
 André Monteiro GOMES & SP3024059 \\
 Bianca Kaori HNG & SP3022455 \\
 Luiz Henrique de Almeida e ALBUQUERQUE & SP3030199 \\
 Natan da Fonseca LISBOA & SP3024784 \\
 Patrícia Santos PASCHOAL & SP3022218 \\
%Aluno 1  & prontuario1 \\
%ALUNO 2 com sobrenome grande & prontuario2 \\
%ALUNO 3 & prontuario3 \\
%ALUNO 4 & prontuario4 \\
\end{tabular}
}


\tipotrabalho{Projeto da Disciplina PI1A5}

\disciplina{PI1A5 - Projeto Integrado I}

\preambulo{Proposta de projeto para disciplina PI1A5}

\data{2021}

% Definir o que for necessário e comentar o que não for necessário
% Utilizar o Nome Completo, abntex tem orientador e coorientador
% então vão ser utilizados na definição de professor
\renewcommand{\orientadorname}{Professor:}
\orientador{DANIEL MARQUES GOMES DE MORAIS}



% ---


% ---
% Configurações de aparência do PDF final


% informações do PDF
\makeatletter
\hypersetup{
        %pagebackref=true,
        pdftitle={\@title}, 
        pdfauthor={\@author},
        pdfsubject={\imprimirpreambulo},
        pdfcreator={LaTeX with abnTeX2},
        pdfkeywords={abnt}{latex}{abntex}{abntex2}{trabalho acadêmico}, 
        colorlinks=true,            % false: boxed links; true: colored links
        linkcolor=blue,             % color of internal links
        citecolor=blue,             % color of links to bibliography
        filecolor=magenta,              % color of file links
        urlcolor=blue,
        bookmarksdepth=4
}
\makeatother
% --- 

% ---

% ----
% Início do documento
% ----
\begin{document}

\frenchspacing 

\pretextual

% ---
% Capa - Para proposta a folha de rosto é suficiente pois é mais completa.
% ---
\imprimirfolhaderosto
% ---

% ----------------------------------------------------------
% ELEMENTOS TEXTUAIS
% ----------------------------------------------------------
\textual

\chapter{Justificativa}
A participação dos alunos nas aulas e atividades escolares é essencial para a evolução do seu aprendizado. Porém, é possível observar que apenas a rotina de horas de aulas combinado a um grande volume de atividades tende a se tornar maçante para a maioria dos alunos, resultando na falta de motivação para desempenhar suas obrigações escolares.

Para contornar essa situação, é possível utilizar elementos que permitam tornar o processo de aprendizado mais interessante para os alunos.  Aplicar conceitos de gamificação na rotina escolar é um deles. O termo gamificação compreende a utilização de elementos de jogos em atividades de não jogos. 

Embora este termo seja recente, a gamificação já tem sido aplicada há muito tempo. Um exemplo são as estrelas recebidas como recompensa quando a criança realiza um bom trabalho.

Hoje em dia, já é possível encontrar plataformas educacionais online que aplicam o conceito de gamificação para promover a aprendizagem do aluno. Entretanto, ao analisar as soluções existentes no mercado, percebe-se que a maioria possui uma deficiência em comum, não ser código aberto. Este tipo de aplicação possui conteúdo engessado e é fortemente atrelado ao fabricante (vendor lock-in). 

No contexto educacional brasileiro, sabe-se que embora as escolas sejam obrigadas a seguir uma grade fixa de disciplinas e conteúdos de base, cada instituição possui sua maneira única de ofertá-lo aos seus alunos. Deste modo, surgiu o interesse de desenvolver uma solução que aplique conceitos de gamificação na educação mas que ao mesmo tempo seja flexível às necessidades dos diferentes clientes. 

Ademais, partindo de uma base de dados que será alimentada ao longo da utilização da aplicação, será possível desenvolver visões gerenciais que permitam aos diretores, pedagogos e professores acompanharem a evolução dos seus alunos, bem como diagnosticarem possíveis problemas e tomarem decisões para superá-los.
\chapter{Objetivo}
Este projeto visa desenvolver o sistema Turma de Elite, uma aplicação de gerenciamento de aprendizado que visa integrar conceitos de gamificação ao ensino, e deste modo se tornar uma ferramenta que busca auxiliar todo corpo docente e discente de uma instituição escolar.

Para o corpo docente, a ferramenta contará com funcionalidades que permitam acompanhar o desenvolvimento do aluno e modelar desafios, atividades e recompensas.

Para o corpo discente, a aplicação implementará funcionalidades que buscam promover o engajamento dos alunos nas aulas, acrescentando ao processo de realizar exercícios e avaliações, uma dinâmica semelhante aos jogos atuais onde ao completar um desafio, ganham-se recompensas, aumenta-se de nível e desbloqueia novos desafios.

Outro objetivo a ser alcançado será desenvolver a ferramenta que seja flexível às diferentes escolas e que também seja configurável ao ponto de se adaptar tanto para o ensino presencial quanto o remoto. 

Inicialmente o sistema Turma de Elite será desenvolvido para atender a turmas do ensino fundamental II, que compreendem alunos do 6º ao 9º ano, com faixa etária variando entre 11 a 15 anos. Posteriormente, a aplicação poderá ser adaptada para abranger o ensino primário e médio.
\chapter{Escopo}
A proposta de projeto é uma aplicação web voltada para auxiliar no acompanhamento da evolução no rendimento dos alunos nas atividades escolares.

Considerando a rotina básica das escolas de ensino fundamental no Brasil, a aplicação contará com o módulo de cadastros que permitirá que o usuário (coordenador, diretor ou pedagogo) realize o cadastro de turmas, disciplinas, gestores e alunos.

Além da configuração básica, o gestor poderá atribuir os professores e alunos às turmas e disciplinas previamente cadastradas. Ademais, deverá configurar a estrutura de conquistas e os parâmetros para que sejam alcançados pelos alunos.

Ao acessar a aplicação, o professor poderá visualizar as turmas e disciplinas atribuídas a ele. Para cada turma/disciplina ele poderá visualizar o ranking dos alunos, inserir e receber as atividades deles.

Quando o aluno acessar a aplicação, o mesmo poderá visualizar as atividades pendentes e o painel de conquistas que apresentará as conquistas alcançadas e não alcançadas por ele. Ademais, poderá visualizar o ranking da turma, sua posição em relação aos seus amigos e a pontuação necessária para que ele suba para a liga (tiers) imediatamente mais alta.

A aplicação também fornecerá um dashboard que conterá gráficos com visões gerenciais que indicam o desempenho por turma e por aluno. Esse dashboard estará disponível apenas para  usuários com o perfil gestor.
As funcionalidades a serem implementadas estão descritas com mais detalhes nos subcapítulos a seguir. 

\section{Funcionalidades}
\subsection{Login e perfis}
O módulo de acesso permitirá que usuários cadastrados acessem o sistema; todos os processos relacionados à definição e reinicialização de senha estão contemplados neste módulo. Também será possível definir, para um determinado perfil, quais são as suas permissões de acesso.
Inicialmente, contempla-se nesta proposta três categorias de perfil de acesso: 

\begin{itemize}
\item Perfil aluno: poderá acessar as disciplinas referentes à turma que ele foi inserido, painel de atividades, painel de conquistas, sua posição no ranking e os alunos que ocupam os três primeiros lugares de cada liga (tiers).
\item Perfil professor: poderá visualizar as turmas atribuídas a ele e o ranking da turma. Será responsável por postar as atividades e ao corrigi-las, atribuir a pontuação para os alunos.
\item Perfil gestor: poderá visualizar todas as turmas cadastradas, o ranking de cada uma delas e o dashboard. Será responsável por realizar os cadastros e parametrizações do sistema.
\end{itemize}

\subsection{Cadastros/Dados mestres}
Segundo os objetivos do projeto, são listados abaixo os cadastros básicos, essenciais ao funcionamento do sistema em questão. Para cada cadastro, o sistema deve permitir a inserção, listagem, alteração e exclusão (ou inativação) de registros. 
\begin{itemize}
\item Usuários: contempla os dados dos usuários que acessarão o sistema (gestores, professores e alunos).
\item Turmas/disciplinas: contempla os dados das turmas/disciplinas da escola;
\item Conquistas: contempla os dados das conquistas. Ao cadastrar uma conquista, o gestor deverá definir a recompensa pela sua conquista, o seu nível e os critérios para alcançá-la.
\item Atividades: contempla os dados das atividades. Uma atividade poderá gerar um entregável ou não.
\end{itemize}

\subsection{Parametrização de turmas}
O gestor deverá utilizar esta funcionalidade para atribuir a cada turma, o professor responsável e os alunos. Para cada turma poderá ser atribuído um único professor. No caso de uma disciplina compartilhar mais de um professor, o gestor deverá criar uma disciplina para a mesma turma para cada professor e dividir os alunos entre elas. 
\subsection{Painel de turmas/disciplinas}
Este painel será exibido assim que o usuário acessar aplicação. Ele consiste em uma grade de botões que darão acesso ao ambiente da disciplina. Este painel estará disponível para as três visões previstas neste projeto, porém as turmas a serem listadas serão restringidas da seguinte forma:
\begin{itemize}
\item Visão do aluno: listará somente as disciplinas da turma no qual foi inserido.
\item Visão do professor: listará somente as turmas/disciplinas atribuídas a ele.
\item Visão do gestor: listará todas as turmas/disciplinas cadastradas.
\end{itemize}

\subsection{Módulo de atividades}
Este módulo será disponibilizado no ambiente da disciplina e será utilizado por professores e alunos. Através dele, o professor poderá postar as atividades e atribuir notas. Ao postar uma atividade o professor poderá informar a pontuação e o prazo de entrega. Caso a atividade possua um entregável, o professor deverá indicar também no momento do cadastro.

Para os alunos, ao acessar esta seção, ele poderá visualizar as atividades postadas pelo professor. Ao realizar uma atividade que necessita de um entregável, o aluno deverá realizar o upload de um arquivo no formato especificado pelo professor na descrição da atividade, após a confirmação da entrega atividade ganhará um status “Pendente de avaliação”. Somente após a avaliação do professor, a atividade será marcada como concluída.

As atividades criadas que não geram entregáveis está ligada ao ensino presencial na qual, ao passar uma atividade na sala, ou um dever de casa, o aluno mostrará a apostila para o professor, e ele dará o seu “visto” pela aplicação. 
\subsection{Painel de conquistas}
O painel de conquista estará disponível para o aluno. Cada aluno terá seu próprio painel de conquistas, e nele o aluno poderá visualizar as conquistas atingidas, as pendentes e as bloqueadas. 

O aluno poderá alcançar somente as conquistas atreladas a liga que ele se encontra e as imediatamente abaixo. A cada liga, as conquistas ficam mais difíceis de alcançar, porém as recompensas serão maiores. 
\subsection{Ranking de alunos por liga}
O sistema de ranqueamento do sistema Turma de Elite contemplará três ligas, sendo elas: bronze, prata e ouro. Cada liga conterá um ranking disponibilizado em duas visões (ranking por disciplina e ranking geral). Sempre ao final de um período pré-determinado, os três primeiros colocados de um ranking que possuem a pontuação mínima requerida pela liga superior, ganham um lugar na próxima liga, enquanto os três últimos descem para uma liga inferior.

Assim como o painel de turmas/ disciplina, os rankings estarão disponíveis para todos os usuários, porém de forma restringida dependendo do perfil.
\begin{itemize}
\item Visão do aluno: visualizará apenas os rankings referente à turma no qual o aluno está inserido. Em cada ranking o aluno saberá apenas o nome dos três primeiros colocados de cada liga e a sua posição caso ele não esteja entre os primeiros.
\item Visão do professor: visualizará os rankings das turmas atribuídas a ele. Em cada ranking, o professor poderá ver todos os alunos e suas respectivas posições.
\item Visão do gestor: visualizará os rankings se todas as turmas ativas na aplicação. Em cada ranking, o gestor poderá ver todos os alunos e suas respectivas posições.
\end{itemize}

\subsection{Dashboard}
A aplicação também disponibilizará relatórios gerenciais para o perfil gestor como: histórico de desempenho por turma e por aluno.

\chapter{Análise comparativa}
Tomando uma plataforma como a Khan Academy, que utiliza conceitos de gamificação para a execução de atividades, nota-se que ela não utiliza o conceito de \textit{tiers} para ranquear os alunos, por exemplo. Esse conceito compreende ao agrupamento de estudantes em ligas diferentes de acordo com o desempenho.

Outro diferencial em relação ao Khan Academy, são as funcionalidades de customização de atividades, parametrização de turmas e conquistas que permitem que a Turma de Elite se adapte não só a diferentes escolas como também as categorias de ensino remoto e à distância.

Uma outra opção possível para gamificar o ensino é a incorporação de um plugin que ofereça ferramentas de gamificação a um sistema de LMS já consolidado no mercado. Ao analisar as funcionalidades de um plugin de gamificação \textit{open-source} chamado "Level Up!", nota-se que ele não contempla algumas funcionalidades para sua utilização no contexto de gestão de uma instituição de ensino, como a visão de gestor, por exemplo.

Além dessas plataformas, foi feita uma análise comparativa com algumas outras. A síntese dos resultados obtidos pode ser representada por meio da tabela abaixo:

\begin{table}[h]

\ABNTEXfontereduzida
\caption{Análise comparativa entre as plataformas de gestão de aprendizado}
\label{tabela-correta-equipamento}
\begin{tabular}[h]{|m{2.0cm}|m{1.5cm}|m{1.5cm}|m{1.5cm}|m{1.5cm}|m{1.5cm}|m{1.5cm}|m{1.5cm}|m{1.5cm}}
\hline
{\thead{}} & \thead{Khan\\ Academy} & \thead{Academy\\ LMS} & \thead{Axonify} & \thead{Matrix \\LM} & 
\thead{Talent \\ LMS} & 
\thead{Moodle \\+ Level\\ Up!} &
\thead{Turma\\ de \\elite} \\ \hline
    Código aberto               &   &   &   &   & X & X & X               \\ \hline
    Customização de atividades  &   & X & X & X & X & X & X               \\\hline
    Medalhas                    & X & X &   & X &   & X & X               \\ \hline
    Tiers / ligas               &   &   &   &   &   &   & X               \\ \hline
    Leaderboards                & X & X & X & X & X & X & X               \\ \hline
    Visão do Gestor             &   &   &   & X &   &   & X \\ \hline   \end{tabular}
\legend{Fonte: Os autores}

\end{table}
\vspace{600mm}
\chapter{Evoluções previstas}
Para próxima release do sistema Turma de Elite já está previsto, além de melhorias, desenvolvimento de novas funcionalidades que serão elencadas nos tópicos a seguir:
\begin{itemize}
\item  Criação do novo perfil "Pais e responsáveis": melhoria no módulo de acesso, o novo perfil permitirá que os pais ou responsáveis pelo aluno acompanhe as atividades do aluno.
\item  Boletins: nova funcionalidade que permitirá que os professores insiram as notas das provas ao fim de cada bimestre.
\item  Plano de aulas: nova funcionalidade que permitirá que o professor realize seu planejamento de aulas. Este plano de aulas poderá ser convertido para um relatório que estará disponível para download para os gestores da escola;
\item  Atividades programadas: melhoria que permitirá que os professores programem a postagem de atividades para seus alunos.
\item  Notificações e Lembretes: melhoria que permitirá que os usuários sejam notificados por e-mail a respeito de ações tomadas na aplicação como, por exemplo, confirmação de envio de atividade.
\end{itemize}
Além das funcionalidades listadas, também está previsto o estudo da viabilidade de uma integração com o LMS Moodle, que será descrito no capítulo a seguir.
\chapter{Integrações}
Tendo em vista que já existe a adoção de sistemas gerenciadores de aprendizado por parte das instituições de ensino, foi decidido como um teste de viabilidade futuro a integração da aplicação com um LMS pronto. O LMS teria a responsabilidade de lidar com toda a parte referente à entrega de atividades pelos alunos. O LMS escolhido foi o Moodle, devido aos seguintes motivos:
\begin{itemize}
    \item{ser código aberto;}
    \item{ter a possibilidade de integração com documentação oficial para tal;} 
    \item{alto uso (187000 sites registrados no mundo e 9437 sites registrados no Brasil, segundo o site da plataforma).}
\end{itemize}

A forma de integração testada inicialmente será utilizando Web Services, onde as operações de gravação e leitura de dados na plataforma será feita através de requisições utilizando o protocolo HTTP.
\chapter{Tecnologia e infraestrutura}
\section{Back-end}
O back-end da aplicação será desenvolvida com a Linguagem Java apoiado pelo framework Spring e seus componentes. 

\section{Front-end}
Para o front-end será utilizado Angular, que é um framework baseado em TypeScript voltado para aplicações Web.

\section{Sistema gerenciador de banco de dados}
Para armazenamento de dados foi escolhido o sistema de gerenciamento de banco de dados MySQL.

\section{Versionamento}
Durante o desenvolvimento, serão utilizados o GitHub e o SVN para o versionamento de código.

\section{Hospedagem da aplicação}
O back-end da aplicação será hospedado no Heroku, utilizando como plataforma de CI, o Travis. O front-end da aplicação será hospedado com o serviço Firebase Hosting, do Google.

\section{Autenticação}
Para dar suporte à autenticação de usuário no sistema, será utilizado o Firebase Authentication.
\chapter{Parcerias e viabilidade comercial}
Inicialmente, o sistema terá como clientes escolas particulares de ensino fundamental, porém com o objetivo de crescimento e expansão, ao ponto de ser oferecido para grandes sistemas de ensino, como Pitágoras, Objetivo e Anglo.

\section{Monetização}
\subsection{Uso do servidor do fornecedor}
Por meio do aluguel da hospedagem fornecida pela empresa fabricante, a instituição de ensino contratante pagará um valor por mês para obter todo o sistema pré-configurado, com manutenção inclusa.

\subsection{Uso de servidor próprio}
Caso a empresa contratante opte por utilizar um servidor próprio, a implementação será feita via pagamento de um valor fixo mais uma taxa de manutenção, em troca de uma implementação customizada que pode incluir configurações únicas, como a personalização da logomarca, por exemplo.

\subsection{Código aberto}
Há ainda a possibilidade da utilização da aplicação de forma gratuita, por ela ser de código aberto: as instituições interessadas podem utilizar tal código da maneira que preferirem, desde que respeitem a licença adotada. Como a licença utilizada será a MIT License, estará garantida a cópia e distribuição da aplicação. Entretanto, eventuais danos e usos indevidos da mesma não serão de responsabilidade da empresa fabricante, que não prestará suporte.

% ----------------------------------------------------------
% Referências bibliográficas
% ----------------------------------------------------------
\bibliography{referencias, abntex2-doc-abnt-6023}

\end{document}