\chapter{Parcerias e viabilidade comercial}
Inicialmente, o sistema terá como clientes escolas particulares de ensino fundamental, porém com o objetivo de crescimento e expansão, ao ponto de ser oferecido para grandes sistemas de ensino, como Pitágoras, Objetivo e Anglo.

\section{Monetização}
\subsection{Uso do servidor do fornecedor}
Por meio do aluguel da hospedagem fornecida pela empresa fabricante, a instituição de ensino contratante pagará um valor por mês para obter todo o sistema pré-configurado, com manutenção inclusa.

\subsection{Uso de servidor próprio}
Caso a empresa contratante opte por utilizar um servidor próprio, a implementação será feita via pagamento de um valor fixo mais uma taxa de manutenção, em troca de uma implementação customizada que pode incluir configurações únicas, como a personalização da logomarca, por exemplo.

\subsection{Código aberto}
Há ainda a possibilidade da utilização da aplicação de forma gratuita, por ela ser de código aberto: as instituições interessadas podem utilizar tal código da maneira que preferirem, desde que respeitem a licença adotada. Como a licença utilizada será a MIT License, estará garantida a cópia e distribuição da aplicação. Entretanto, eventuais danos e usos indevidos da mesma não serão de responsabilidade da empresa fabricante, que não prestará suporte.