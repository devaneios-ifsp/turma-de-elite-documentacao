\chapter{Justificativa}
A participação dos alunos nas aulas e atividades escolares é essencial para a evolução do seu aprendizado. Porém, é possível observar que apenas a rotina de horas de aulas combinado a um grande volume de atividades tende a se tornar maçante para a maioria dos alunos, resultando na falta de motivação para desempenhar suas obrigações escolares.

Para contornar essa situação, é possível utilizar elementos que permitam tornar o processo de aprendizado mais interessante para os alunos.  Aplicar conceitos de gamificação na rotina escolar é um deles. O termo gamificação compreende a utilização de elementos de jogos em atividades de não jogos. 

Embora este termo seja recente, a gamificação já tem sido aplicada há muito tempo. Um exemplo são as estrelas recebidas como recompensa quando a criança realiza um bom trabalho.

Hoje em dia, já é possível encontrar plataformas educacionais online que aplicam o conceito de gamificação para promover a aprendizagem do aluno. Entretanto, ao analisar as soluções existentes no mercado, percebe-se que a maioria possui uma deficiência em comum, não ser código aberto. Este tipo de aplicação possui conteúdo engessado e é fortemente atrelado ao fabricante (vendor lock-in). 

No contexto educacional brasileiro, sabe-se que embora as escolas sejam obrigadas a seguir uma grade fixa de disciplinas e conteúdos de base, cada instituição possui sua maneira única de ofertá-lo aos seus alunos. Deste modo, surgiu o interesse de desenvolver uma solução que aplique conceitos de gamificação na educação mas que ao mesmo tempo seja flexível às necessidades dos diferentes clientes. 

Ademais, partindo de uma base de dados que será alimentada ao longo da utilização da aplicação, será possível desenvolver visões gerenciais que permitam aos diretores, pedagogos e professores acompanharem a evolução dos seus alunos, bem como diagnosticarem possíveis problemas e tomarem decisões para superá-los.