\chapter[Introdução]{Introdução}
A participação dos alunos nas aulas e atividades escolares é essencial para a evolução do aprendizado. Porém, é possível observar que apenas a rotina de horas de aula combinada a um grande volume de atividades com poucas recompensas imediatas tendem a fazer com que os estudos se tornem maçantes para a maioria dos alunos, resultando na falta de motivação para desempenhar suas obrigações escolares.


No contexto educacional brasileiro, sabe-se que embora as escolas sejam obrigadas a seguir uma grade fixa de disciplinas e conteúdos de base, cada instituição possui sua maneira única de ofertá-lo aos seus alunos. Deste modo, surgiu o interesse de desenvolver uma solução que aplique conceitos de \textit{gamificação} na educação para motivar os alunos, mas que ao mesmo tempo seja flexível às necessidades dos diferentes clientes. 


Ademais, partindo de uma base de dados que será alimentada ao longo da utilização da aplicação, será possível desenvolver visões gerenciais que permitam aos diretores, pedagogos e professores acompanharem a evolução dos seus alunos, bem como diagnosticarem possíveis problemas e tomarem decisões para superá-los.

\section{Objetivos}
Este projeto visa desenvolver o sistema Turma de Elite, uma aplicação de gerenciamento de aprendizado que tem como objetivo integrar conceitos de \textit{gamificação} ao ensino, e deste modo se tornar uma ferramenta que busca auxiliar todo corpo docente e discente de uma instituição escolar.


Para o corpo docente, a ferramenta contará com funcionalidades que permitam acompanhar o desenvolvimento do aluno e modelar desafios, atividades e recompensas.


Para o corpo discente, a aplicação implementará funcionalidades que buscam promover o engajamento dos alunos nas aulas, acrescentando ao processo de realizar exercícios e avaliações, uma dinâmica semelhante aos jogos atuais onde ao completar um desafio, ganham-se recompensas, aumenta-se de nível e desbloqueia novos desafios.

De modo geral, o projeto Turma de Elite tem a missão de promover um maior interesse dos alunos nos estudos por meio do aumento de fatores motivacionais na realização de tarefas, utilizando a \textit{gamificação} como elemento auxiliador nesse processo.

Tendo em vista a inserção tecnológica ocorrendo cada vez mais cedo na sociedade, coloca-se como objetivo secundário a maior disseminação da tecnologia no ambiente estudantil, de maneira benéfica.

\section{Justificativa}
Atualmente já é possível encontrar plataformas educacionais online que aplicam o conceito de \textit{gamificação} para promover a aprendizagem do aluno. Entretanto, ao analisar as soluções existentes no mercado, percebe-se que a maioria possui uma deficiência em comum, não possuir código aberto. Este tipo de aplicação faz com que o usuário da tecnologia proprietária sofra o chamado aprisionamento tecnológico, onde o usuário se torna dependente do fornecedor, não podendo trocar de fornecedor, se necessário, sem um custo adicional considerável, portanto o usuário perde liberdade, flexibilidade e escalabilidade (\textit{\gls{vendor-lock-in}}). Por isso, esta aplicação será de código aberto, com isso, o usuário poderá transferir suas informações para outro provedor ou para um provedor próprio por um baixo custo.


\section{Análise de concorrência}
Entre os principais concorrentes diretos do sistemas pode-se citar:

\begin{itemize}

\item {\textit{Khan Academy}:} utiliza conceitos de \textit{gamificação} para a execução de atividades, nota-se que ela não utiliza o conceito de \textit{\glspl{tier}} para ranquear os alunos, por exemplo. Esse conceito compreende o agrupamento de estudantes em ligas diferentes de acordo com o desempenho. Outro diferencial da aplicação Turma de Elite em relação ao \textit{Khan Academy}, são as funcionalidades de customização de atividades, parametrização de turmas e conquistas que permitem que a Turma de Elite se adapte não só a diferentes escolas como também às categorias de ensino remoto e à distância;

\item {\textit{Academy LMS}:} possui suporte para múltiplas visões de usuário.
Entretanto, como ela não possui código aberto, uma forma de obtenção de lucro encontrada foi a cobrança por meio de uma assinatura;

\item{\textit{Axonify}:} possui algumas funcionalidades que podem ser interessantes para os alunos no que diz respeito à implementação de \textit{gamificação}, como pontos, conquistas, medalhas e \textit{\glspl{ranking}}. Entretanto, tal solução também não possui código aberto, nem visão para um administrador;

\item{\textit{Matrix LMS}:} aparenta ser a solução mais completa, uma vez que possui todas as visões de usuário que a Turma de Elite pretende implementar (aluno, professor e gestor). Entretanto, uma grande desvantagem dela é o custo, uma vez que, além de uma assinatura a ser cobrada para continuar utilizando a plataforma, os materiais oferecidos também são pagos;

\item{\textit{Talent LMS}:} pode-se constatar algumas ausências importantes para um sistema de aprendizagem gamificado, como a premiação com medalhas. Apesar disso, esse sistema compensa tal ausência com a implementação de pontos e níveis;


\item{\textit{Moodle} + \textit{Level Up}!:} uma outra opção possível para gamificar o ensino é a incorporação de um \textit{\gls{plugin}} que ofereça ferramentas de gamificação a um sistema de \textit{\ac{lms}} já consolidado no mercado. Ao analisar as funcionalidades de um \textit{\gls{plugin}} de \textit{gamificação} \textit{\gls{open-source}} chamado \textit{"Level Up!"}, nota-se que ele não contempla algumas funcionalidades para sua utilização no contexto de gestão de uma instituição de ensino, como a visão de gestor, por exemplo.

\end{itemize}

Para uma melhor visualização, o \autoref{quadro-analise-comparativa} contém a comparação sintetizada de todos os sistemas analisados.

\begin{quadro}[htb]
\centering
\ABNTEXfontereduzida
\caption{\label{quadro-analise-comparativa}Análise comparativa entre as plataformas de gestão de aprendizado}
\begin{tabular}{|m{2.3cm}|m{1.8cm}|m{1.8cm}|m{1.5cm}|m{1.5cm}|m{1.2cm}|m{1.5cm}|m{1.3cm}|m{1.8cm}}
\hline
{\thead{}} & \thead{Khan\\ Academy} & \thead{Academy\\ LMS} & \thead{Axonify} & \thead{Matrix \\LMS} & 
\thead{Talent \\ LMS} & 
\thead{Moodle \\+ Level\\ Up!} &
\thead{Turma\\ de \\elite} \\ \hline
    Código aberto               &   &   &   &   &   & X & X               \\ \hline
    Customização de atividades  &   & X & X & X & X & X & X               \\\hline
    Medalhas                    & X & X & X & X & X & X & X               \\ \hline
    Tiers / ligas               &   &   &   &   &   &   & X               \\ \hline
    Leaderboards                & X & X & X & X & X & X & X               \\ \hline
    Visão do Gestor             &   &   &   & X &   &   & X \\ \hline   
\end{tabular}
\fonte{Os autores}
\end{quadro}