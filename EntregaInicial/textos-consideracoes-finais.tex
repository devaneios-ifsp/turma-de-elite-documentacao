\chapter[Considerações Finais]{Considerações Finais}

Com a elaboração da documentação e do desenvolvimento da aplicação até o \ac{mvp}, os integrantes da equipe tiveram a oportunidade de tomar conhecimento da real complexidade que é um desenvolvimento de \textit{\gls{software}}. Para tal, algumas dificuldades ao longo do desenvolvimento ocorreram. 

A primeira e principal dificuldade que todos os integrantes passaram foi o tempo. Como a maioria da equipe já trabalha na área e também houve uma condensação do semestre, isso reduziu a quantidade de horas que poderia ser dedicado ao projeto. Além disso, como havia outras disciplinas com tarefas a serem entregues, com um semestre condensado também, tornava mais difícil conciliar essas entregas com as expectativas do andamento do projeto.

A segunda dificuldade encontrada foi a falta de conhecimento prévio em algumas tecnologias e ferramentas adotadas. Com isso, acabava-se gastando o pouco tempo disponível para buscar saber sobre o assunto. Para tentar minimizar essa dificuldade, nas reuniões semanais, era apresentada a tecnologia em questão, uma visão geral de como funcionava e também \textit{links} para facilitar o entendimento e procura sobre a tecnologia.

A terceira dificuldade foi a utilização de tecnologias obrigatórias pela disciplina. O LaTeX, por exemplo, é uma ferramenta que demandou um grande esforço, pois, mesmo que alguns integrantes da equipe já sabiam utilizar, sempre aparecia uma situação nova para resolver, então era necessário pesquisar e testar diversas soluções para identificar e solucionar o problema. O Overleaf, editor do LaTeX adotado, apresentou algumas instabilidades no decorrer da escrita da documentação da aplicação, ocasionadas por manutenção. Entretanto, tais instabilidades ocorreram raramente e foram resolvidas rapidamente. Além disso, o \gls{svn} também demandou um certo esforço, pois nenhum dos integrantes tinha conhecimento dessa ferramenta.

No caso do gerenciamento do projeto, a principal dificuldade encontrada foi a comunicação entre os integrantes da equipe, afinal era aparente que nem todos estavam satisfeitos e por dentro do andamento do projeto. Então, para minimizar isso, foi criado um documento no \textit{Notion} para que todos colocassem suas atividades do dia, próximos passos e possíveis impedimentos, visando que todos tomassem ciência do andamento do projeto e poder ajudar caso alguém estivesse precisando. Além disso, foi decidido também que a cada duas semanas, todos os integrantes iriam falar o que estavam sentindo em relação ao projeto, para assim todos terem voz e o sentimento de pertencimento dentro da equipe.

Outra dificuldade encontrada no gerenciamento foi lidar com diferentes pessoas, com diferentes formas de pensar e de empenhar-se ao projeto. Foi necessário colocar ao limite o senso crítico para sempre buscar tomar a decisão certa e assim encontrar a melhor saída para possíveis conflitos e desencontros que aconteciam, desde definir os próximos passos para cada integrante até para alinhá-los com as expectativas do desenvolvimento e entregas do projeto.

As dificuldades encontradas ao longo do projeto serviram para o crescimento dos integrantes, bem como a identificação de melhorias para o futuro. Além disso, os conhecimentos adquiridos poderão ser levados não apenas para o desenvolvimento do produto pronto, mas também para o ambiente profissional e pessoal em diversos aspectos.

