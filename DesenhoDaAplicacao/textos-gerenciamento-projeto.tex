\chapter{Gerenciamento do Projeto}
Este capítulo visa definir a parte de gestão de projeto, incluindo a metodologia adotada, a organização da equipe e a gestão do tempo.

\section{Metodologia de Gestão}
A metodologia de gerenciamento de projeto adotada é baseada no Scrum.

O Scrum é um \textit{\gls{framework}} para gestão de projetos que tem como principais objetivos agilizar o processo de desenvolvimento do \textit{\gls{software}} e responder rapidamente às mudanças. Ela é composta por cerimônias e tem como um dos pilares as \textit{\glspl{sprint}}, ou seja, um conjunto de atividades para um determinado tempo. 


As atividades de uma \textit{\gls{sprint}} serão planejadas em uma \textit{Sprint Planning}, que é a primeira cerimônia e tem como principal objetivo a definição das entregas para o final da \textit{\gls{sprint}}. Nessa cerimônia, baseando-se nas histórias priorizadas no \textit{Product Backlog}, será criado o \textit{Sprint Backlog} em um arquivo em Excel, que contém as atividades e suas principais informações, como o responsável, o status, podendo ser \textit{To Do} (a fazer), \textit{Doing} (fazendo) e \textit{Done} (feito), e a data de conclusão. 


Após o planejamento da \textit{\gls{sprint}}, a execução das atividades será iniciada. A \textit{\gls{daily}} será feita através do aplicativo \textit{\gls{notion}}, no qual cada um colocará as atividades feitas no dia, os próximos passos e os impedimentos, caso houver. Além disso, reuniões de acompanhamento serão feitas para ver o andamento do projeto e alinhamento das expectativas da \textit{\gls{sprint}}. Desse modo, esses \textit{\glspl{checkpoint}} semanais acontecerão nas terças-feiras às 19:30 ao longo da sprint. 

No último dia da \textit{\gls{sprint}}, às 19h30, será feita uma reunião para revisão das entregas, a chamada \textit{Sprint Review}, na qual será identificado o que foi entregue com sucesso, o que não foi concluído e também possíveis mudanças no \textit{\gls{product-backlog}}. 


Após a \textit{Sprint Review}, será feita uma reunião de retrospectiva, a \textit{Sprint Retrospective}, para avaliar o desempenho da equipe, o andamento do projeto e possíveis melhorias para a próxima \textit{\gls{sprint}}. A \textit{Sprint Planning} da \textit{\gls{sprint}} seguinte será realizado logo após a \textit{Sprint Review} da anterior.


Todas as reuniões acontecerão através do \textit{Google Meet}.
Para eventuais dúvidas, alinhamentos e possíveis urgências será utilizado o \textit{WhatsApp} e, se necessário, reuniões no \textit{Google Meet}. 


\section{Organização da Equipe}
As atividades foram divididas entre os integrantes da equipe segundo as habilidades e interesses de cada um, levando-se em consideração também o nível de dificuldade de cada frente.


Desse modo, o André será o \textit{\gls{tech-lead}} da equipe e focará na frente de desenvolvimento do sistema, portanto a parte de preparação do ambiente, desenvolvimento \textit{\gls{back-end}} e \textit{\gls{front-end}} serão as suas principais atividades, mas também atuará na elaboração de documentação, na fase inicial do projeto, e como suporte, ao longo dele.


A Bianca será a gerente de projetos e \textit{\gls{scrum-master}} da equipe, sendo a responsável por organizar a equipe e por garantir as entregas do projeto. Será responsável também pela postagem no blog semanalmente e auxiliará na elaboração da documentação e na criação dos vídeos. E poderá auxiliar nas outras frentes conforme houver necessidade.


O Luiz fará parte do time de desenvolvimento e auxiliará no desenvolvimento \textit{\gls{back-end}}, sobretudo na parte de testes, e \textit{\gls{front-end}}, na estilização das telas, mas também atuará, principalmente, na frente da documentação. Ele auxiliará também na parte dos vídeos e poderá desempenhar outras funções conforme houver necessidade.


O Natan também fará parte do time de desenvolvimento e auxiliará no desenvolvimento do \textit{\gls{back-end}}, na parte de testes e no desenvolvimento da aplicação em si, mas também focará na documentação. Será o principal responsável pela criação, edição e postagem dos vídeos e também poderá atuar nas outras frentes conforme houver necessidade.


A Patrícia será a \textit{\gls{product-owner}} e terá como principal foco a documentação, principalmente a elicitação de requisitos e definição do escopo. Ela cuidará também da frente dos Dados da aplicação e auxiliará no \textit{\gls{front-end}}, sobretudo na parte de protótipos de telas. E poderá auxiliar nas outras frentes conforme houver necessidade.


\section{Gestão de Tempo}

Baseando-se no Scrum, a gestão do tempo será feita pelas \textit{\glspl{sprint}}. Cada \textit{\gls{sprint}} terá a duração de duas semanas e, no total, serão quinze \textit{\glspl{sprint}}, com previsão de datas de início e de fim conforme descrito na Tabela 2. 


\ABNTEXfontereduzida
\begin{table}[htb]
\centering
\caption{Data de início e data fim de cada \textit{sprint}}
\begin{tabular}{|c|c|c|}
   \hline
   \thead{Sprint} & \thead{Data Início}  & \thead{Data Fim}   \\\hline
    1 & 10/05/21 & 25/05/21 \\\hline
    2 & 25/05/21 & 08/06/21 \\\hline
    3 & 08/06/21 & 22/06/21 \\\hline
    4 & 22/06/21 & 06/07/21 \\\hline
    5 & 06/07/21 & 20/07/21 \\\hline
    6 & 20/07/21 & 03/08/21 \\\hline
    7 & 03/08/21 & 17/08/21 \\\hline
    8 & 17/08/21 & 31/08/21 \\\hline
    9 & 31/08/21 & 14/09/21 \\\hline
    10 & 14/09/21 & 28/09/21 \\\hline
    11 & 28/09/21 & 12/10/21 \\\hline
    12 & 12/10/21 & 26/10/21 \\\hline
    13 & 26/10/21 & 09/11/21 \\\hline
    14 & 09/11/21 & 23/11/21 \\\hline
    15 & 23/11/21 & 14/12/21 \\\hline
\end{tabular}
\fonte{Os autores}
\end{table}
\FloatBarrier


As atividades priorizadas para cada \textit{\gls{sprint}} serão baseadas no cronograma da disciplina e serão agrupadas visando a sua finalização até o final da \textit{\gls{sprint}}, mas dependendo da dificuldade da atividade, ela poderá ser iniciada em uma \textit{\gls{sprint}} e terminar em outra.