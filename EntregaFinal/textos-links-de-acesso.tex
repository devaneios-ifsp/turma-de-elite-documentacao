\section{Links de acesso}

Os \textit{\glspl{link}} de acesso pertinentes ao projeto estão disponíveis nesta seção.

A \autoref{qr-github} contém o \textit{\gls{qr-code}} que leva ao principal repositório remoto utilizado para o versionamento de código durante o desenvolvimento da aplicação, o \textit{GitHub}.
A fim de obter uma melhor divisão lógica, foram criados dois repositórios na plataforma para hospedar os códigos relacionados ao \textit{\gls{front-end}} e ao \textit{\gls{back-end}} do projeto.

\begin{figure}[htb]
\begin{flushright}
\begin{pspicture}(25mm,25mm)
\psbarcode{https://github.com/devaneios-ifsp}{eclevel=H width=1.0 height=1.0}{qrcode}
\end{pspicture}
\caption{\label{qr-github}\textit{QR-Code} - Repositório remoto GitHub}
\legend{\url{https://github.com/devaneios-ifsp}}
\fonte{Os Autores}
\end{flushright}
\end{figure}
\FloatBarrier


Além do GitHub, o repositório Subversion também foi utilizado para gerenciamento do código produzido, sendo atualizado periodicamente. A \autoref{qr-svn} contém o \textit{\gls{qr-code}} do repositório em questão.

\begin{figure}[htb]
%\begin{flushright}
\begin{pspicture}(25mm,25mm)
\psbarcode{https://svn.spo.ifsp.edu.br/viewvc/A6PGP/S202101-PI/DevAneios/}{eclevel=H width=1.0 height=1.0}{qrcode}
\end{pspicture}
\caption{\label{qr-svn}\textit{QR-Code} - Repositório remoto Subversion}
\legend{\url{https://svn.spo.ifsp.edu.br/viewvc/A6PGP/S202101-PI/DevAneios/}}
\fonte{Os Autores}
%\end{flushright}
\end{figure}
\FloatBarrier

Todos os vídeos produzidos pela equipe podem ser acessados no canal do Youtube da mesma, cujo \gls{link} está contido no \textit{\gls{qr-code}} da \autoref{qr-yt}.

\begin{figure}[htb]
\begin{flushright}
\begin{pspicture}(25mm,25mm)
\psbarcode{https://www.youtube.com/channel/UCjwYXnZHuCg74AkhR5aZs6A/featured}{eclevel=H width=1.0 height=1.0}{qrcode}
\end{pspicture}
\caption{\label{qr-yt}\textit{QR-Code} - Canal DevAneios no Youtube}
\legend{\url{https://www.youtube.com/channel/UCjwYXnZHuCg74AkhR5aZs6A/featured}}
\fonte{Os Autores}
\end{flushright}
\end{figure}
\FloatBarrier

Os relatórios semanais de atividades realizadas por cada membro do projeto são publicadas no blog da equipe, cujo \gls{link} pode ser acessado no \textit{\gls{qr-code}} da \autoref{qr-blog}.

\begin{figure}[htb]
%\begin{flushright}
\begin{pspicture}(25mm,25mm)
\psbarcode{https://devaneiosifsp.blogspot.com/}{eclevel=H width=1.0 height=1.0}{qrcode}
\end{pspicture}
\caption{\label{qr-blog}\textit{QR-Code} - Blog DevAneios}
\legend{\url{https://devaneiosifsp.blogspot.com/}}
\fonte{Os Autores}
%\end{flushright}
\end{figure}
\FloatBarrier

A página de login e cadastramento do site Turma de Elite pode ser acessada através do {\gls{qr-code}} na \autoref{qr-login}, com as credenciais informadas.

\begin{figure}[htb]
\begin{flushright}
\begin{pspicture}(25mm,25mm)
\psbarcode{https://turma-de-elite-app.web.app/login}{eclevel=H width=1.0 height=1.0}{qrcode}
\end{pspicture}
\caption{\label{qr-login}\textit{QR-Code} - Tela de Login Turma de Elite}
\legend{\url{https://turma-de-elite-app.web.app/login}}
\fonte{Os Autores}
\end{flushright}
\end{figure}
\FloatBarrier

Para acessar a aplicação, usar o seguinte e-mail e senha:
\begin{itemize}
    \item E-mail: admteste6@gmail.com
    \item Senha: 123456
\end{itemize}

Os detalhes da API do projeto são documentados na aplicação Swagger. Seu \gls{link} de acesso pode ser encontrado através do {\gls{qr-code}} da \autoref{qr-swagger}.

\begin{figure}[htb]
%\begin{flushright}
\begin{pspicture}(25mm,25mm)
\psbarcode{https://turma-de-elite.herokuapp.com/swagger-ui/}{eclevel=H width=1.0 height=1.0}{qrcode}
\end{pspicture}
\caption{\label{qr-swagger}\textit{QR-Code} - Swagger do Turma de Elite}
\legend{\url{https://turma-de-elite.herokuapp.com/swagger-ui/}}
\fonte{Os Autores}
%\end{flushright}
\end{figure}
\FloatBarrier
