% ---
% RESUMOS
% ---
% resumo em português
\setlength{\absparsep}{18pt} % ajusta o espaçamento dos parágrafos do resumo
\begin{resumo}

 \vspace{\onelineskip}

 A \textit{gamificação} consiste no uso de mecânicas e características de jogos em atividades que, inicialmente, não aplicam os elementos dos jogos. O principal objetivo desse trabalho é apresentar a aplicação Turma de Elite, um sistema web de gestão de aprendizado que implementa conceitos de \textit{gamificação} e possui como principal público-alvo os estudantes das redes de ensino, contemplando, em primeira instância, os alunos do 6$^\circ$ ao 9$^\circ$ ano do ensino fundamental, para depois abranger os outros níveis do ambiente escolar. Propõe-se, desse modo, criar uma alternativa tecnológica que promova um maior engajamento dos estudantes no processo de aprendizado, de modo que eles tenham uma maior motivação para adquirir conhecimento por meio dos estudos. Para o desenvolvimento da aplicação, foi utilizado Angular para o front-end, Firebase Authentication para a autenticação dos dados do usuário, servidor SMTP da Google para envio de e-mails de confirmação, o serviço S3 da Amazon para armazenar arquivos enviados pelos usuários, PostgreSQL como banco de dados, Spring Boot para criação e configuração da aplicação, Spring Data para auxiliar no acesso aos dados do banco de dados e Heroku para a hospedagem dos serviços. Além disso, serão abordados neste documento questões como o escopo e gerenciamento do projeto, bem como todo processo de produção do mesmo, considerando levantamentos e descartes efetuados. 
 
 
 \vspace{\onelineskip}
 
 \textbf{Palavras-chave}: Gamificação. Aprendizado. Engajamento. Angular. Spring Boot. Heroku.
 
\end{resumo}

% resumo em inglês
\begin{resumo}[Abstract]
 \begin{otherlanguage*}{english}
 
    \vspace{\onelineskip}
 
     Gamification consists in the use of mechanics and games features in activities that, initially, does not apply the game concepts. The main purpose of this paper is present the Turma de Elite application, a web Learning Management System that implements gamification concepts and have the education network students as the main target audience, contemplating, in the first instance, 6th to 9th elementary school students, to then cover the other education levels.
     Thus, it is proposed to create a technological alternative that leads to greater engagement of the students in the learning process, so that they would be more motivated to acquire knowledge through their studies. The application was develop using Angular for the front-end, Firebase Authentication for users login data authentiacation, Google´s SMTP server for sending confirmation emails, Amazon's S3 service, to storage files sent by the users, PostegreSQL as a data base, Spring Boot for creating and configuring the application, Spring Data to facilitate data base access and Heroku for hosting the application. In addition, issues such as the scope and management of the project will be addressed in this document, as well as its entire production process, considering surveys and disposals made.
     
    \todo[inline]{fazer tradução do resumo, não utilizar tradução automática}
   \vspace{\onelineskip}
   \noindent 
   \textbf{Keywords}: Gamification. Learning. Engagement. Angular. Spring Boot. Heroku.
 \end{otherlanguage*}
\end{resumo}