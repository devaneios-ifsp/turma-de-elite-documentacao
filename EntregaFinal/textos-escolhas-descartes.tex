\section{Escolhas e Descartes}
No início do projeto, foi decidido que para o produto pronto ficaria apenas a integração com o Moodle. Entretanto, para complementá-la, foi necessário ter novas ideias. Uma das ideias que surgiram foi inserir um sistema de boletim na aplicação, entretanto foi logo descartada, pois o Moodle já realiza essa função. Por fim, foi decidido que também ficaria para a próxima entrega os dashboards, além de trazer responsividade para a aplicação. Além disso, ao longo do desenvolvimento, algumas funcionalidades acabaram não fazendo mais sentido ao longo do processo como o envio de e-mail ao gestor para cadastrar dados da sua escola. Ao invés disso, é possível cadastrar seus dados diretamente pelo administrador. 

Outra mudança que também ocorreu foi a utilização do PostgreSQL, ao invés do MySQL, para o armazenamento de dados. Foi uma recomendação do professor orientador Daniel Marques Gomes de Morais, visto que o PostgreSQL é um sistema gerenciador de banco de dados que tem integração com o Heroku, portanto não gera custos adicionais de hospedagem. Com essa mudança, a utilização do Oracle Cloud também foi descartada, visto que ele era usado apenas para hospedar o MySQL e não possuía integração gratuita com o \ac{sgbd}.

Foi alterado também a forma da abordagem da metodologia ágil no projeto. Antes, havia muitas ferramentas, como o Excel e o Notion, que eram atualizadas para seguir o Scrum, mas com as dicas dos professores da banca, foi possível centralizar em uma única ferramenta, o Projects do GitHub, e assim ter uma melhor visualização e fácil alteração dos status das atividades. Além disso, com essa ferramenta também é possível conciliar as tarefas aos seus respectivos códigos, facilitando a identificação do andamento, bem como a geração de métricas do projeto.
