%% Adaptado a partir de :
%%    abtex2-modelo-trabalho-academico.tex, v-1.9.2 laurocesar
%% para ser um modelo para os trabalhos no IFSP-SPO

\documentclass[
    % -- opções da classe memoir --
    12pt,               % tamanho da fonte
    openright,          % capítulos começam em pág ímpar (insere página vazia caso preciso)
    %twoside,            % para impressão em verso e anverso. Oposto a oneside
    oneside,
    a4paper,            % tamanho do papel. 
    % -- opções da classe abntex2 --schwinn
    % Opções que não devem ser utilizadas na versão final do documento
    %draft,              % para compilar mais rápido, remover na versão final
    paginasA3,  % indica que vai utilizar paginas em A3 
    MODELO,             % indica que é um documento modelo então precisa dos geradores de texto
    TODO,               % indica que deve apresentar lista de pendencias 
    % -- opções do pacote babel --
    english,            % idioma adicional para hifenização
    brazil              % o último idioma é o principal do documento
    ]{ifsp-spo-inf-ctds} % ajustar de acordo com o modelo desejado para o curso



% ---
% Informações de dados para CAPA e FOLHA DE ROSTO
% ---
\titulo{AVALIAÇÃO DAS OUTRAS EQUIPES - ENTREGA INICIAL}

% Trabalho em Equipe
% ver também https://github.com/abntex/abntex2/wiki/FAQ#como-adicionar-mais-de-um-autor-ao-meu-projeto
\renewcommand{\imprimirautor}{
\begin{tabular}{lr}
     André Monteiro Gomes & SP3024059 \\
     Bianca Kaori Hng & SP3022455\\
     Gustavo Manoel Santos* & SP3022391 \\
     Luiz Henrique de Almeida e Albuquerque & SP3030199\\
     Natan da Fonseca Lisboa & SP3024784\\
     Patrícia Santos Paschoal & SP3022218\\
     \\
     *Membro da equipe DevAneios a partir de 05/07/2021
\end{tabular}
}

\disciplina{PI1A5 - Projeto Integrado I}

\preambulo{Trabalho apresentado ao Instituto Federal de Educação, Ciência e Tecnologia de São Paulo - Câmpus São Paulo - como parte dos requisitos para aprovação na disciplina Projeto Integrado I (PI1A5), do curso superior de Tecnologia em Análise e Desenvolvimento de Sistemas.}

\data{2021}

% Definir o que for necessário e comentar o que não for necessário
% Utilizar o Nome Completo, abntex tem orientador e coorientador
% então vão ser utilizados na definição de professor
\renewcommand{\orientadorname}{Professor:}
\orientador{DANIEL MARQUES GOMES DE MORAIS}



% ---


% informações do PDF
\makeatletter
\hypersetup{
        %pagebackref=true,
        pdftitle={\@title}, 
        pdfauthor={\@author},
        pdfsubject={\imprimirpreambulo},
        pdfcreator={LaTeX with abnTeX2 using IFSP model},
        pdfkeywords={abnt}{latex}{abntex}{abntex2}{IFSP}{\ifspprefixo}{trabalho acadêmico}, 
        colorlinks=true,            % false: boxed links; true: colored links
        linkcolor=blue,             % color of internal links
        citecolor=blue,             % color of links to bibliography
        filecolor=magenta,              % color of file links
        urlcolor=blue,
        bookmarksdepth=4
}
\makeatother

% ----
% Início do documento
% ----
\begin{document}

% Retira espaço extra obsoleto entre as frases.
\frenchspacing 

% -- lista de pendencias gerada pelo todonotes
% -- altere opções do usepackage para remover na versão final....

% ----------------------------------------------------------
% ELEMENTOS PRÉ-TEXTUAIS
% ----------------------------------------------------------
\pretextual

% ---
% Capa
% ---
\imprimircapa

% ---
% inserir lista de figuras
% ---
\pdfbookmark[0]{\listfigurename}{lof}
\listoffigures*
\cleardoublepage

% ---
% inserir o sumario
% ---
\pdfbookmark[0]{\contentsname}{toc}
\tableofcontents*
\cleardoublepage
% ---


% ----------------------------------------------------------
% ELEMENTOS TEXTUAIS
% ----------------------------------------------------------
\textual

\chapter{Equipe ConsacreTADS}

A equipe ConsacreTADS apresentou como projeto a aplicação Safira, uma aplicação \textit{\gls{mobile}} com o intuito de auxiliar o usuário final no controle de gastos pessoais.

\section{Apresentação}

No geral, os integrantes da equipe se expressaram bem. Entretanto, a apresentação deixou a desejar pelo fato da aplicação não possuir nenhum processo automatizado significativo para controle de gastos, que poderia ser gerado pela conexão a uma conta bancária, por exemplo. Em vez disso, a aplicação possuía somente um processo de registro manual de receitas e despesas.


Unido a esse fato, foi possível notar que a equipe não conseguiu utilizar, de maneira apropriada, as referências sobre \textit{\gls{open-banking}} oferecidas pelo professor da disciplina para incorporar funcionalidades à aplicação, trazendo tais referências de maneira muito superficial na apresentação.

\section{Documentação}
Infelizmente, a documentação da equipe nesta entrega deixou muito a desejar, uma vez que, em primeiro lugar, o modelo de documento proposto pelo Instituto Federal de São Paulo não foi utilizado, fazendo com que o conteúdo não ficasse formatado de acordo com os padrões da instituição.


Além disso, foi possível notar que a equipe não se atentou muito às exigências da disciplina, pois faltaram vários elementos requeridos para essa entrega da documentação.

\section{Sugestões de melhorias}
Em primeiro lugar, sugere-se a leitura atenta do \textit{website} que contém todos os requisitos da disciplina. O \textit{\gls{link}} de acesso à pagina pode ser acessado pelo \textit{\gls{qr-code}} da \autoref{qr-dicas-do-ivan}.

\begin{figure}[htb]
\begin{flushright}
\begin{pspicture}(25mm,25mm)
\psbarcode{https://dicas.ivanfm.com/aulas/pi1a5.html}{eclevel=H width=1.0 height=1.0}{qrcode}
\end{pspicture}
\caption{\label{qr-dicas-do-ivan}\textit{QR-Code} - Dicas do Ivan - Projeto Integrado I}
\legend{\url{https://dicas.ivanfm.com/aulas/pi1a5.html}}
\fonte{Os Autores}
\end{flushright}
\end{figure}
\FloatBarrier

Outra sugestão é a elaboração de uma análise de concorrência mais refinada com o objetivo de estabelecer melhor as funcionalidades do \textit{\gls{software}}, tendo como base outras soluções já consolidadas no mercado. Além disso, tal análise também pode ser útil para elaborar diferenciais mais chamativos para o usuário final, de modo que ele prefira utilizar a aplicação Safira para realizar o controle de gastos pessoal ao invés de fazê-lo manualmente por uma planilha.




\input{pos-glossario}

\end{document}