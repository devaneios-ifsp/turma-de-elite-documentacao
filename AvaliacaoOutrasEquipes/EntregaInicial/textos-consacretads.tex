\chapter{Equipe ConsacreTADS}

A equipe ConsacreTADS apresentou como projeto a aplicação Safira, uma aplicação \textit{\gls{mobile}} com o intuito de auxiliar o usuário final no controle de gastos pessoais.

\section{Apresentação}

No geral, os integrantes da equipe se expressaram bem. Entretanto, a apresentação deixou a desejar pelo fato da aplicação não possuir nenhum processo automatizado significativo para controle de gastos, que poderia ser gerado pela conexão a uma conta bancária, por exemplo. Em vez disso, a aplicação possuía somente um processo de registro manual de receitas e despesas.


Unido a esse fato, foi possível notar que a equipe não conseguiu utilizar, de maneira apropriada, as referências sobre \textit{\gls{open-banking}} oferecidas pelo professor da disciplina para incorporar funcionalidades à aplicação, trazendo tais referências de maneira muito superficial na apresentação.

\section{Documentação}
Infelizmente, a documentação da equipe nesta entrega deixou muito a desejar, uma vez que, em primeiro lugar, o modelo de documento proposto pelo Instituto Federal de São Paulo não foi utilizado, fazendo com que o conteúdo não ficasse formatado de acordo com os padrões da instituição.


Além disso, foi possível notar que a equipe não se atentou muito às exigências da disciplina, pois faltaram vários elementos requeridos para essa entrega da documentação.

\section{Sugestões de melhorias}
Em primeiro lugar, sugere-se a leitura atenta do \textit{website} que contém todos os requisitos da disciplina. O \textit{\gls{link}} de acesso à pagina pode ser acessado pelo \textit{\gls{qr-code}} da \autoref{qr-dicas-do-ivan}.

\begin{figure}[htb]
\begin{flushright}
\begin{pspicture}(25mm,25mm)
\psbarcode{https://dicas.ivanfm.com/aulas/pi1a5.html}{eclevel=H width=1.0 height=1.0}{qrcode}
\end{pspicture}
\caption{\label{qr-dicas-do-ivan}\textit{QR-Code} - Dicas do Ivan - Projeto Integrado I}
\legend{\url{https://dicas.ivanfm.com/aulas/pi1a5.html}}
\fonte{Os Autores}
\end{flushright}
\end{figure}
\FloatBarrier

Outra sugestão é a elaboração de uma análise de concorrência mais refinada com o objetivo de estabelecer melhor as funcionalidades do \textit{\gls{software}}, tendo como base outras soluções já consolidadas no mercado. Além disso, tal análise também pode ser útil para elaborar diferenciais mais chamativos para o usuário final, de modo que ele prefira utilizar a aplicação Safira para realizar o controle de gastos pessoal ao invés de fazê-lo manualmente por uma planilha.


